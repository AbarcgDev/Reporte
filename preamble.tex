\documentclass[12pt, letterpaper]{article}
\usepackage{url}
\usepackage[utf8]{inputenc}
\usepackage[table,xcdraw]{xcolor}
\usepackage{geometry}

\usepackage{pgfplots}
\pgfplotsset{compat=1.18}

% Configuración de márgenes según APA
\geometry{
  left=1in,
  right=1in,
  top=1in,
  bottom=1in,
}

% Configuración de la fuente y espaciado
\usepackage{mathptmx} % Times New Roman para el texto
\usepackage[scaled=.92]{helvet} % Arial para el sans-serif
\usepackage{setspace}
\setstretch{2} % Doble espaciado según APA

% Otros paquetes necesarios
\usepackage[spanish,es-nodecimaldot,es-tabla]{babel}
\usepackage{graphicx}
\usepackage{tikz}
\usepackage{pgf}
\usepackage{tocloft}
\graphicspath{{./figs/}}
\usepackage{comment}
\usepackage{hyperref}
\usepackage{titling}
\usepackage{bookmark}
\usepackage{listings}
\usepackage{mathtools} % Times New Roman para el texto
\urlstyle{same}

% Configuración de enlaces y colores
\hypersetup{
   colorlinks=true,
   urlcolor=blue, % Cambiado a azul según APA
   linkcolor=black,
   citecolor=black,
   filecolor=magenta,
   pdfpagemode=FullScreen,
}

% Encabezado y pie de página
\usepackage{fancyhdr}
\pagestyle{fancy}
\fancyhf{} % Limpiar todos los encabezados y pies de página
\renewcommand{\headrulewidth}{0.4pt}
\setlength{\headheight}{15pt} % Ajustar la altura del encabezado

% Configuración específica del encabezado
\fancyhead[C]{\footnotesize} % Título del trabajo a la derecha
\fancyhead[L]{\footnotesize\leftmark} % Capítulo o sección en el centro
\fancyhead[R]{\footnotesize\thepage} % Número de página a la izquierda

% Configuración específica del pie de página
\fancyfoot[R]{\footnotesize\theauthor} % Grupo a la derecha
\fancyfoot[C]{\footnotesize} % Sin información en el centro
\fancyfoot[L]{\footnotesize\asignatura} % Asignatura a la izquierda
\renewcommand{\footrulewidth}{0.4pt} % Línea del pie de página

% referencias
\usepackage{natbib}
\bibliographystyle{apalike} % Estilo de citas y referencias APA

% Ajustes adicionales para listados y otros
\usepackage{minted}
\usemintedstyle{friendly}
\definecolor{backcolour}{gray}{0.95}
\lstdefinestyle{mystyle}{
    backgroundcolor=\color{backcolour},
    basicstyle=\ttfamily\footnotesize,
    breakatwhitespace=false,
    breaklines=true,
    captionpos=b,
    keepspaces=true,
    numbers=left,
    numbersep=5pt,
    showspaces=false,
    showstringspaces=false,
    showtabs=false,
    tabsize=2
}
\lstset{style=mystyle}

% Otros ajustes de formato según tus necesidades específicas
% Local Variables:
% TeX-command-extra-options: "-shell-escape"
% End:
