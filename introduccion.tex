\label{sec:introduccion}
Se realizó un experimento para mostrar una aplicación de la integración numérica, específicamente por el método de la Regla Compuesta de Simpson.
El experimento consiste en la realización de un sistema de adquisición de datos conformado, principalmente, de un sensor de flujo de corriente basado
en el efecto Hall modelo ACS712 y, para el procesado de las lecturas de este sensor, se implementó un sencillo firmware en un microcontrolador ESP32.
Los datos obtenidos por este sistema se procesaron para obtener una medición de la potencia generada por un dispositivo electrónico durante veinticuatro horas
de uso. Con base en las leyes de funcionamiento de la corriente alterna, al calcular la integral de la potencia generada con respecto al tiempo se obtuvo la cantidad de energía consumida por el dispositivo durante el periodo de medición.

\clearpage
